\title{Final Project for Digital Signal Processing: COSC390}
\author{Dr. Jordan Hanson - Whittier College Dept. of Physics and Astronomy}
\date{\today}
\documentclass[10pt]{article}
\usepackage[a4paper, total={18cm, 27cm}]{geometry}
\usepackage{outlines}
\usepackage{hyperref}

\begin{document}
\maketitle

\begin{abstract}
The final project will involve data brought to the class by the individual students.  The data must be digitized or digital in nature, and be analyzed to establish a result to be presented to the class.  The final presentation should be longer, approximately 15 minutes.  Students should begin gathering and/or sharing data early on in the course to facilitate creation of the final presentation.
\end{abstract}
\noindent
\textit{\textbf{Format}: The goal of the project is to analyze digital data to reveal some effect or test a hypothesis.  The data should be digital in nature, but may be from any field of math, science, or social science.  Time should be spent in the presentation on the hypothesis being tested, or data query being posed.  Second, time should be spent on explaining how the data was obtained, and how it was cleaned or edited.  Third, the methods of analysis will need to be explained, using either equations or diagrams.  Finally, the results need to be shown in figures or tables, and the validity of the hypothesis revealed.} \\
\textit{\textbf{Grading}: For attention to detail in the presentation: 20\%.  Examples of attention to detail are correctly labeling axes, giving proper units to quantities even if they are digital, and correctly explaining a common algorithm or procedure.  Accuracy and precision of analysis: 20\%.  An example of accuracy and presion in this context means plotting a spectrum with the correct maximum frequency, i.e. error avoidance.  Clear expression of ideas: 40\%.  A presentation has little value if no one can understand it.  Emphasis should be given to clear expressions of the ideas, with fewer words and slower pace instead of a fast pace with lots of words.  Use of ideas learned in class: 20\%.  Examples would be the correct use of an FFT, or the correct use of a filter or response function.  If the presentation is on audio processing, for example, use of a concept from chapter 22 of the book would be useful.} \\ \\
\textbf{Example presentation outline: Approximately 15 minutes.  \textit{Speak slowly and carefully.}}
\begin{outline}[enumerate]
\1 Slide 1: Introduction to audio processing
\1 Slide 2: Audio Processing a sound pattern
\2 A sound was collected from the following source...
\2 What can we learn about the sound via the FFT algorithm?
\2 A spectrum of a tone should be concentrated around a single frequency.
\1 Slide 3: What is an FFT
\1 Slide 4: How the sounds were collected
\2 Equipment
\2 Setup
\1 Slide 5: How the data was processed
\2 Step 1: Getting the data onto my laptop
\2 Step 2: Selecting the data sample
\1 Slide 6: How the FFT was used on this data
\2 Plot of the FFT magnitude
\2 Plot of the FFT phase
\2 State any methods use to eliminate noise
\1 Slide 7: Hypothesis comparison
\2 The spectrum was concentrated around the correct frequency
\2 There were other frequencies that had power
\2 There was noise
\1 Slide 8: Conclusion

\end{outline}
\end{document}
