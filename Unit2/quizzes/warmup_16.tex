\documentclass{article}
\usepackage{graphicx}
\usepackage[margin=1.5cm]{geometry}
\usepackage{amsmath}

\begin{document}
\twocolumn

\title{Thursday Warm Up, Unit 1: Filter Design}
\author{Prof. Jordan C. Hanson}
\maketitle

\section{The Moving Average Filer}

\begin{enumerate}
\item Implement the following code in \verb+octave+, to create a gaussian pulse with a width, $\sigma$, and frequency, $f$:
\begin{verbatim}
fs = 40e3;
dt =1/fs;
T = 1.0;
t = dt:dt:T;
f = 440.0;
sigma = 0.5;
t0 = 0.5;
T = 2.0;
A = 4.0;
x = A*cos(2*pi*f*t).*exp(-0.5*(t-t0).^2/sigma^2);
\end{verbatim}
Create a player to play this data as audio:
\begin{verbatim}
player1 = audioplayer(x,fs,16);
play(player1)
\end{verbatim}
\item Add random gaussian noise, to create a noisy signal with an SNR of 5.  Create a new player to play it as audio:
\begin{verbatim}
y = x+rand(size(x))*0.2;
player2 = audioplayer(y,fs,16);
play(player2)
\end{verbatim}
How does the audio sound, compared to the noiseless version?
\item The \textbf{moving average filter} is the \textit{optimal solution} for reducing random gaussian noise, while preserving step response.  Implement the moving average filter by defining the kernel like a normalized step:
\begin{verbatim}
kern = ones(1,400)/400;
z = conv(kern,y);
player3 = audioplayer(z,fs,16);
play(player3)
\end{verbatim}
\item Play \verb+player1+ and \verb+player3+ repeatedly.  Do they sound similar?  If \verb+player3+ sounds quieter, try amplifying it by increasing the signal amplitude.  That is, multiply the samples by a constant, and feed it back to the audioplayer algorithm.  At a certain point, do you notice \textit{clipping}, or distortion from overamplifying?
\end{enumerate}

\section{Graphing the Results}

\begin{enumerate}
\item \textbf{Graph} the noiseless version of the gaussian pulse, with appropriate units, below: \\ \vspace{5cm}
\item \textbf{Graph} the noisy version of the gaussian pulse, with appropriate units, below: \\ \vspace{5cm}
\item \textbf{Graph} the filtered version of the gaussian pulse, with appropriate units, below: \\ \vspace{5cm}
\end{enumerate}

\end{document}
