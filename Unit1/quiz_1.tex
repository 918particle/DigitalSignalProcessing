\documentclass{beamer}
\usetheme{metropolis}
\usepackage{graphicx}
\usepackage{amsmath}
\title{Digital Signal Processing: COSC360}
\author{Jordan Hanson}
\institute{Whittier College Department of Physics and Astronomy}

\begin{document}
\maketitle

\section{Complex numbers 1: theory and examples}

\begin{frame}{Complex numbers 1}
Convert each of these complex numbers to polar form:
\begin{enumerate}
\item $z = 4 + 4j$
\item $z = 1$, $z = j$, $z = -1$, $z = -j$
\item In the previous problem, describe in words what is happening to the \textit{phase angle} of each number.
\end{enumerate}
Convert each of these complex numbers to rectangular form ($z = x + jy$).
\begin{enumerate}
\item $z = 2 \exp(j \pi/4)$
\item $z = 5 \exp(j \pi)$
\end{enumerate}
\end{frame}

\begin{frame}{Complex numbers 1}
Suppose that $x_i = 2\pi ft+\phi_i$.  The sum of two sinusoids \textit{in the complex plane} with amplitudes $a_1$ and $a_2$ can then be written
\begin{equation}
V(t) = a_1\exp(j x_1) + a_2\exp(j x_2)
\end{equation}
It is assumed that we would take the real part of $V(t)$ to be physical.
\begin{enumerate}
\item Compute $|V|^2 = V^*V$, and $\phi_2 - \phi_1 = \pi$, $\phi_2 - \phi_1 = 0$.
\item What is $\phi_V = \tan^{-1}(\operatorname{Im}\lbrace V \rbrace/\operatorname{Re}\lbrace V \rbrace)$ in each case?
\end{enumerate}
Why do these results make sense?  Thus, the complex numbers encapsulate the concepts of \textit{constructive} and \textit{destructive} interference.
\end{frame}

\section{Complex numbers 3: Application to AC cicuits}

\begin{frame}{Complex numbers 1: application to AC circuits}
\small
Recall the RLC circuit transfer function looks like
\begin{align}
h(\omega) &= \frac{Z_2 + Z_3}{Z_1+Z_2+Z_3} \\
\omega_{LC}^{-2} &= LC \\
\tau &= RC \\
k^2 &= 1-\left(\frac{\omega}{\omega_{LC}}\right)^2 \\
h(\omega) &= \frac{k^4}{k^4+(\omega\tau)^2}-j \frac{k^2\omega\tau}{k^4+(\omega\tau)^2}
\end{align}
and
\begin{align}
Z_R &= R + 0i \\
Z_C &= 0 + \frac{1}{j\omega C} \\
Z_L &= 0 + j \omega L
\end{align}
\end{frame}

\begin{frame}{Complex numbers 1: application to AC circuits}
Recompute $h(\omega)$, but start with $L = 0$ ($Z_2 = 0$).  This reduces the circuit to an RC circuit.  It is still useful to have $\tau = RC$.  Draw a graph of $|h(\omega)|$.
\end{frame}


\end{document}
