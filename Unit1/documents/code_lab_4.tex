\title{Code Lab 4: Fourier Transforms, Spectra, and the FFT}
\author{Prof. Jordan C. Hanson}
\date{\today}

\documentclass[10.5pt]{article}
\usepackage[margin=2cm]{geometry}
\usepackage{amsmath}
\usepackage{graphicx}
\usepackage{hyperref}

\begin{document}
\twocolumn
\maketitle

\begin{abstract}
In this activity, we create a digitized, sampled sinusoidal signal, plus noise, and compute its Fourier transform using the \verb+fft+ algorithm in \verb+octave+.  The signal should be visible above the \textit{noise floor.}  We will tune the various DSP parameters, and eliminate noise with low and high-pass filter algorithms.
\end{abstract}

\section{The Full Code}

For the full code, download the Code Lab 4 file from the course Moodle page.  The title of the code is \verb+FFT.m+.  Run this code to create a graph of a \textit{noisy sine wave} in the time and frequency domains.  The Fourier transform magnitude is plotted in the lower graph.  The \verb+fft+ algorithm produces positive and negative frequencies.  Both are plotted, and the data should be similar.

\section{Fine Tuning Parameters}

Do not worry if you do not understand every aspect of the code.  Our job here is to locate the following parameters in the code, and tune them to different values:

\begin{enumerate}
\item \verb+fs+, the sampling frequency.
\item \verb+f_sig+, the sine wave frequency.
\item \verb+amplitude+, the number of modes in the signal spectrum.
\item \verb+noise_sigma+, the standard deviation of our gaussian thermal noise.
\end{enumerate}

\section{Exercises}

\begin{enumerate}
\item Change the sine wave frequency, and observe how the single frequency-mode moves around the spectrum.
\item Reset your parameters, and increase the noise so that the sine wave is not visible in the \textit{time domain}, but remains visible in the \textit{frequency domain.}
\item Reset your parameters, and copy the code from line 19 for the sine wave.  Past it back in line 19 so that the signal \verb+y+ is equal to a sine-squared plus noise.  Where is the frequency mode?  Use trigonometric identities to show that the frequency of the sine-squared is the sum of the frequencies of the individual sine waves.
\item Reset your parameters, and activate the low and high-pass filters by uncommenting lines 22-25.  Look up how the \verb+butter+ function works by running the \verb+help butter+ command in octave.
\item The arguments to the filters are how many ``poles'' the low and high-pass filters have, and where the cutoff frequency is located.  The cutoff frequency is expressed as a fraction of the Nyquist frequency.
\item \textbf{Tuning the filters:} Tune the filter parameters to eliminate noise, except for the spectral region around the sine wave signal.
\item \textbf{Draw a graph} of your filtered signal output from \verb+octave+ below, including both the \textit{time domain} and \textit{frequency domain} data:
\end{enumerate}

\end{document}