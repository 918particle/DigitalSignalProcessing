\title{Asynchronous Activity 1: Radio Waves and Radar}
\author{
Prof. Jordan C. Hanson
}
\date{\today}

\documentclass[12pt]{article}
\usepackage[margin=2cm]{geometry}
\usepackage{amsmath}
\usepackage{graphicx}

\begin{document}
\maketitle

\begin{abstract}
This activity contains exercises related to DSP, radio waves, and radar as part of our COSC360 course.  Please watch Course Videos 1.1 and 1.2 before completing the exercises below.
\end{abstract}

\section{Radio Waves and Octave Syntax}

\begin{enumerate}
\item Write a script entitled radioWave.m for Octave that performs the following tasks:
\begin{itemize}
\item Defines a sample time $\Delta t$ in seconds
\item Defines a maximum time $T_{max}$ in seconds
\item Defines a frequency $f$ in Hz
\item Defines a signal amplitude $A$ in Volts
\item Creates a vector of times running from  0 to $T_{max}$, with each time $\Delta t$ seconds after the previous time.  For example: \verb+t = [0:dt:Tmax]+.
\item Creates a graph of $y1 = A\cos(2\pi f .* t)$, with labeled axes
\end{itemize}
\item Suppose this signal represents an observation of a radio wave on an antenna.  Given what you know about radio waves, what is the wavelength $\lambda$?
\item In Octave, type \verb+help circshift+.  Once you understand how the \verb+circshift+ function works, use it to make a copy of your $y1$ data above, but shifted in time, and add it to the graph. (Remember that the command \verb+hold on+ will allow you to keep adding to the graph without losing anything).
\item If the earlier wave represents the original signal, and the later wave represents a \textit{radar echo}, how far away is the object that created the echo?
\item Finally, make a copy of $y2$ in a variable called $y3$, but change the amplitude to something lower by a factor of 100:
\begin{equation}
y3 = y2/100;
\end{equation}
What is the ratio of the power of signal $y3$ to $y2$ in dB? (Refer to Video 1.2 for a demonstration).
\end{enumerate}

\end{document}