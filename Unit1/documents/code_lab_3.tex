\title{Code Lab 3: Signals, Noise, and Aliasing}
\author{Prof. Jordan C. Hanson}
\date{\today}

\documentclass[12pt]{article}
\usepackage[margin=2cm]{geometry}
\usepackage{amsmath}
\usepackage{graphicx}
\usepackage{hyperref}

\begin{document}
\twocolumn
\maketitle

\begin{abstract}
In this activity, we compare sums of sinusoidal signals and other types of signals and noise.  This connection will be further developed as the course proceeds.  First, we will create a sum of sinusoidal signals at increasing frequencies.  Second, we will plot that sum and compare it to a known type of signal: square pulses.  Third, we will add gaussian white noise to our signals.  Finally, we will plot the \textit{spectrum} of the total signal, and examine how changing the sampling frequency changes the results.
\end{abstract}

\section{The Full Code}

For the full code, download the Code Lab 3 file from the course Moodle page.  The title of the code is \verb+Aliasing.m+.  Run this code to create a graph of a \textit{square wave} in the time and frequency domains, generated from a sum of sinusoids.  The signal has a finite number of modes, or sinusoids at specific frequencies.  The signal has noise present, as well.

\section{Fine Tuning Parameters}

Do not worry if you do not understand every aspect of the code.  Our job here is to locate the following parameters in the code, and tune them to different values:

\begin{enumerate}
\item \verb+fs+, the sampling frequency.
\item \verb+f_0+, the square wave frequency.
\item \verb+n_modes+, the number of modes in the signal spectrum.
\item \verb+noise_sigma+, the standard deviation of our gaussian thermal noise.
\end{enumerate}

\section{Exercises}

\begin{enumerate}
\item Decrease the sampling frequency such that the highest mode in the signal is above the Nyquist frequency (one half of the sampling frequency).  What happens to the signal?
\item Reset your parameters, and tune the square wave frequency higher until the highest mode in the spectrum exceeds the Nyquist frequency.  What happens to the signal?
\item Reset your parameters, and tune the noise level such that it buries some of the high-frequency modes.  What happens to the signal?
\item Reset your parameters, and increase/decrease the number of modes in the signal.  What happens to the signal?
\end{enumerate}

\end{document}