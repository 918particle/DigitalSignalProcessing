\title{Code Lab 1: Audio Signals and Complex Numbers}
\author{Prof. Jordan C. Hanson}
\date{\today}

\documentclass[12pt]{article}
\usepackage[margin=2cm]{geometry}
\usepackage{amsmath}
\usepackage{graphicx}

\begin{document}
\twocolumn
\maketitle

\begin{abstract}
In this activity, we begin to draw a connection between complex numbers and sinusoidal signals.  This connection will be further developed as the course proceeds.  First, we will create a sinusoidal signal at a single frequency, use it to generate audio data, and play the sound.  Second, we will create complex digital data and show that it produces equivalent sounds.
\end{abstract}

\section{Generate Signals at a \\ Single Frequency}

Use the following code to create digitized sinusoidal data in Octave:

\begin{verbatim}
clear;
close;
home;

fs = 44000.0;
dt = 1/fs;
f = 440.0;

t = 0.0:dt:0.1;
S1 = sin(2.0*pi*f.*t);
S2 = sin(2.0*pi*f*2.*t);
\end{verbatim}

The first three lines are standard housekeeping in Octave.  We delete any persistent data or variables with \verb+clear+, and we close any open figures.  We also send the cursor to the top with \verb+home+ if we are working in an interactive Octave environment.  Next, we define a \textit{sampling frequency}, set to 44 kHz.  A Hz is equivalent to one oscillation per second, and 1 Hz is s$^{-1}$.  The variable \verb+dt+ or $\Delta t$ is the time of each sample, 1/fs.  \textbf{Show that $\Delta t$ is about 23 $\mu$s.}  Next, we define an audio frequency, set to 440 Hz.  We then create sampled times between 0 s and 0.1 s, in intervals of $\Delta t$.

The variables \verb+S1+ and \verb+S2+ contain sampled, digitized sine waves with frequencies of $f$ and $2f$, or 440 Hz and 880 Hz.  The following code \textit{concatenates} four copies of each signal, in an alternating sense, using the \verb+mod+ function to check if $i$ is even or odd:

\begin{verbatim}
S_all = [];
for i=1:8
    if(mod(i,2)==0)
        S_all = [S_all S1];
    else
        S_all = [S_all S2];
    endif
endfor
\end{verbatim}
Play the audio using the following code:
\begin{verbatim}
player = audioplayer(S_all,fs,8);
play(player)
\end{verbatim}

\section{Using Complex Exponentials}

Replace the code that creates the signals \verb+S1+ and \verb+S2+ with complex exponentials like $\exp(j \phi(t))$.  Play the \textit{real part} of the signal, then play the \textit{imaginary part}.  What results do you hear?

\end{document}