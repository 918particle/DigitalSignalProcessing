\documentclass{article}
\usepackage{graphicx}
\usepackage[margin=1.5cm]{geometry}
\usepackage{amsmath}

\begin{document}
\twocolumn

\title{Homework 2, Unit 0: Foundations and Fundamentals}
\author{Prof. Jordan C. Hanson}

\maketitle

\section{Memory Bank}
\small
\begin{itemize}
\item $\bar{x} = \frac{1}{N}\sum_{i=0}^{N-1} x_i$ ... Sample mean.
\item $\overline{x^2} = \frac{1}{N}\sum_{i=0}^{N-1} x_i^2$ ... Sample mean of the square.
\item $s = \frac{1}{N-1}\sum_{i=0}^{N-1} (x_i - \bar{x})^2$ ... Sample std. deviation.
\item $s^2 = \overline{x^2} - \overline{x}^2$ ... Formula for the variance.
\item Let a \textbf{histogram} be defined by $M$ bins $i$, with the data organized into $M$ \textit{frequencies} $H_i$.
\item Total number of data points in a histogram: $N = \sum_{i=0}^{M-1} H_i$
\item (1) Sample mean and (2) variance from histograms: 
\begin{enumerate}
\item $\bar{x} = \frac{1}{N}\sum_{i=0}^{M-1} i H_i$
\item $s = \frac{1}{N-1}\sum_{i=0}^{M-1} (i-\bar{x})^2 H_i$
\end{enumerate}
\item For the following two formulas: $\omega = 2\pi f$, $\tau = RC$.
\item \textbf{Low-pass filter response}, as a function of frequency:
\begin{equation}
R(f) = \frac{1}{1+j\omega \tau}
\end{equation}
\item \textbf{High-pass filter response}, as a function of frequency:
\begin{equation}
R(f) = \frac{j\omega\tau}{1+j\omega \tau}
\end{equation}
\end{itemize}
\normalsize

\section{Probability and Statistics, Noise}

\begin{enumerate}
\item Consult Fig. 2-6 in Ch. 2 of the course text. (a) Write three functions in \verb+octave+ that produce the following: a square wave, a triangle wave, and gaussian noise.  (b) Write code that creates histograms of the outputs of the three functions.  (c) \textbf{Normalize} the histograms by dividing the frequencies by the total number of data samples, $N$. (d) Graph the histograms to demonstrate that each process matches the PDFs in Fig. 2-6. (e) Compute the mean and standard deviation of each PDF.\vspace{4cm}\footnote{\textit{Hint: (1) square waves with amplitudes of 0 and 1 should have a mean of 0.5, (2) this is also true of flat PDFs, which also have a standard deviation of $1/\sqrt{12}$, and (3) Eq. 2-6 in the course text gives the Gaussian PDF, which has a std. dev. of $\sigma$.}}
\end{enumerate}

\section{ADC and DAC}

\begin{enumerate}
\item Complete the following exercises about the precision of ADC and DAC components:
\begin{itemize}
\item ADC:
\begin{enumerate}
\item What is the $\Delta V$ (voltage per level) of an ADC with signals in the [0,2.55] V range with 255 levels, plus zero (8-bit precision)? \\ \\
\item What is the $\Delta V$ (voltage per level) of an ADC with signals in the [0,4.095] V range with 4095 levels, plus zero (12-bit precision)? \\ \\
\item How many bits of precision, or how many voltage levels, are required for $\Delta V < 1$ mV, if signals are in the [0,12] V range? \\ \\
\item What is the digital amplitude (in counts) of a 2.52 V signal, if signals are in the [0,5] V range, and there are 2048 levels? \\ \\
\end{enumerate}
\item DAC:
\begin{enumerate}
\item If the digital amplitude of a signal is 256 counts, and signals are in the [0,5] V range with 9.8 mV per level, what is the signal amplitude in volts? \\ \\
\item If the digital amplitude of a signal is 2048 counts, and signals are in the [0,5] V range with max counts 4095, what is the signal amplitude in volts? \\ \\
\item If the digital amplitude of a signal is 128 counts, the max counts is 511, and the analog output is 0.25 V, what is the maximum voltage? \\ \\ \\ \\
\end{enumerate}
\end{itemize}
\item For the following exercises, refer to Fig. 3-4 in Ch. 3 of the course text.
\begin{enumerate}
\item If the sampling rate is 500 kHz, and the analog signal frequency is 50 kHz, what is the digital signal frequency? \\ \\
\item If the sampling rate is 500 kHz, and the analog signal frequency is 250 kHz, what is the digital signal frequency? \\ \\
\item If the sampling rate is 500 kHz, and the analog signal frequency is 750 kHz, what is the digital signal frequency? \\ \\
\item If the sampling rate is 500 kHz, and the analog signal frequency is 1000 kHz, what is the digital signal frequency? \\ \\
\end{enumerate}
\item Consider Fig. 3-10 in the course text.  The single-pole, low-pass RC filter is depicted in the top middle section of Fig. 3-10. (a) Suppose a signal has an amplitude of 3.3 V and a frequency of 25 MHz, while $R=10$ k$\Omega$.  What value of $C$ is necessary to filter the signal to 0.33 V? \\ \vspace{4cm}
\item Consider again Fig. 3-10.  The single-pole, high-pass RC filter is similar to the depiction in the top middle section of Fig. 3-10, but with the $C$ and $R$ switched. (a) Suppose a signal has an amplitude of 3.3 V and a frequency of 10 MHz, while $R=10$ k$\Omega$.  What value of $C$ is necessary to filter the signal to 0.33 V? \\ \vspace{4cm}
\item \textbf{Bonus Point:} What is the phase shift introduced by the filters in the previous two exercises?
\end{enumerate}

\end{document}
