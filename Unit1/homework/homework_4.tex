\documentclass{article}
\usepackage{graphicx}
\usepackage[margin=1.5cm]{geometry}
\usepackage{amsmath}

\begin{document}
\twocolumn

\title{Homework 4, Unit 0: Foundations and Fundamentals}
\author{Prof. Jordan C. Hanson}

\maketitle

\section{Memory Bank}
\small
\begin{itemize}
\item \textbf{Convolution}: this is an operation that characterizes the response $h[n]$ of a linear system.
\begin{equation}
y[i] = h[n] * x[n] = \sum_{j=0}^{M-1}h[j]x[i-j] \label{eq:conv}
\end{equation}
In words, the output at sample $i$ is equal to the produce of the system response $h$ and the input signal $x$, summed over the proceeding $M$ samples (from $j=0$ to $j=M-1$).
\item \textbf{Discrete Delta Function}, $\delta[n]$: A standard impulse response that contains one non-zero sample.  It has the following property:
\begin{equation}
x[n] = \delta[n] * x[n] \label{eq:conv2}
\end{equation}
\item \textbf{Discrete Fourier Transform}, for a sampled, digitized signal $x_n$:
\begin{equation}
X_{\rm k} = \sum_{n=0}^{N-1} x_n e^{-2\pi j (k/N) n} \label{eq:dft}
\end{equation}
\item In DFT analysis, we often need to know the $\Delta t$, time duration for samples, and the sampling rate, $f_{\rm s}$.  Note that $1/f_{\rm s} = \Delta t$.
\item For a sinusoid of frequency $f$ (Hz), the period is $T = 1/f$ (seconds).
\item \textbf{Inverse Discrete Fourier Transform}, for a sampled, digitized signal $X_k$ in the frequency domain:
\begin{equation}
x_{\rm n} = \frac{1}{N} \sum_{k=0}^{N-1} X_k e^{2\pi j (k/N) n} \label{eq:dft2}
\end{equation}
\end{itemize}
\normalsize

\section{Impulse Response}

\begin{enumerate}
\item \textbf{Impulse response of audio echo system.}  Let the sampling frequency be 20 kHz.  (a) Start with a 2-second $\delta[n]$.  How many samples should it contain? (b) Modify the $\delta[n]$ to create an echo every 0.2 seconds, and give the locations of the non-zero samples.  (c) Modify the response function to make each echo half the amplitude as the previous echo. (d) Test your DSP echo on a sine-tone that is 0.1 seconds long.
\item \textbf{Impulse response of a band-pass filter.} Let $l[n]$ and $h[n]$ be the impulse responses of  single-pole low and high pass filters with the same cutoff frequency, $f_{\rm c}$, respectively.  (a) Show that, when an input signal $s[n]$ is split into two copies and sent to $l[n]$ and $h[n]$ \textit{in parallel}, the sum of the outputs is still $s[n]$. (b) Show that the result in (a) implies that $h[n] = \delta[n] - l[n]$. (c) Now assume the cutoff frequencies are different for $h[n]$ and $l[n]$.  If the filters act \textit{in series}, the result is a \textit{band pass} filter, if (choose A, B, C, or D):
\begin{itemize}
\item A: the $f_{\rm c}$ of $l[n]$ is lower than that of $h[n]$.
\item B: the $f_{\rm c}$ of $h[n]$ is lower than that of $l[n]$.
\item C: the $f_{\rm c}$ of $l[n]$ is equal to that of $h[n]$.
\item D: the $f_{\rm c}$ of $l[n]$ and $h[n]$ are equal to one-half the sampling frequency.
\end{itemize} 
A bandpass filter filters data below one cutoff frequency, and above another cutoff frequency, leaving a ``pass band'' in the spectrum.
\end{enumerate}

\section{Discrete Fourier Transform, \\ Filtering, and Noise}

\begin{enumerate}
\item \textbf{Discrete Fourier Transform properties.}  (a) Knowing that the DFT is a complex sum (see Eq. \ref{eq:dft}), prove that the DFT as a DSP operator is additive and homogeneous.  (b) Let $X_k = \delta[k]$ be a frequency-domain signal equal to a constant at the frequency corresponding to $k=k_0$ in Eq. \ref{eq:dft2}, and zero otherwise.  Show that the \textit{inverse} DFT (see Eq.\ref{eq:dft2}) of $\delta[k]$ is a complex sinusoid with frequency $k_0$.  This is one way to demonstrate \textit{sinusoidal fidelity,} that the frequency represented in the time-domain is the same one represented in the frequency domain. \\
\item \textbf{Spectrum of a Square Pulse.} Download the Code Lab 8 (\verb+compare_spectra.m+) from the course Moodle page.  (a) Run the code, and explain in your own words why the magnitude of the Fourier spectrum \textit{widens} as the pulse width \textit{narrows.}  In the figure generated by the code, the Fourier spectra are shown in the left column, while the time-domain signals are shown in the right column. (b) Measure the width of the time-domain signals and the Fourier spectra in a consistent fashion, and show that the product of the time-domain width and Fourier domain width is a constant.  \textit{This is known as the uncertainty principle, that the width of the signal in one domain is inversely proportional to the width in the other domain.}
\end{enumerate}

\end{document}
