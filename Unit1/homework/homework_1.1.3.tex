\documentclass{beamer}
\usetheme{metropolis}
\usepackage{graphicx}
\usepackage{amsmath}
\title{Digital Signal Processing: COSC390}
\author{Jordan Hanson}
\institute{Whittier College Department of Physics and Astronomy}

\begin{document}
\maketitle

\begin{frame}{Homework 1.1.3 - Question 1}
If you have located your data for the final presentation, check to make sure it has less than 100,000 entries.  That is, if it's a $N\times M$ matrix, make sure $N\times M \leq 10^5$.  Some of you have already made suggestions to me, which is great!  Keep them coming.
\end{frame}

\begin{frame}{Homework 1.1.3 - Question 2}
Modify the code Fourier\_Series\_Saw.m (from Moodle, Unit 1 code folder) to produce the 20-term Fourier series of a square wave, as we derived in class.  To remind ourselves:
\begin{align}
A_0 &= 1.0 \\
A_1 &= A_2 = ... 0.0 \\
B_{2n} &= 0 \\
B_{2n+1} &= 2/(n\pi)
\end{align}
In words: the Fourier series of a square wave has all $B_n$ non-zero, with $n$ being and odd integer.  The other terms are all zero except $A_0$, which is 1.0.  \textit{This assignment is due Friday.}  To turn it in, please email me your octave script.
\end{frame}

\end{document}