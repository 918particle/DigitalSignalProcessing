\documentclass{beamer}
\usetheme{metropolis}
\usepackage{graphicx}
\usepackage{amsmath}
\title{Digital Signal Processing: COSC390}
\author{Jordan Hanson}
\institute{Whittier College Department of Physics and Astronomy}

\begin{document}
\maketitle

\begin{frame}{Homework 1.1.2 - Question 1}
Have you started to look for data to analyze for the final presentation?  If not...you should probably start :-)
\end{frame}

\begin{frame}{Homework 1.1.2 - Question 2}
Starting with Eqs. 62, 63, and 64 in lecture notes 1.1, obtain eq. 69 via eq. 65.  Plot the result with the code RLC.m on Moodle in the code folder for Unit 1.  This is an octave script that can be run in the octave environment.  What happens when you tweak the values of $R$, $L$, and $C$?
\end{frame}

\end{document}