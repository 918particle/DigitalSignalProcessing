\documentclass{beamer}
\usetheme{metropolis}
\usepackage{graphicx}
\usepackage{amsmath}
\title{Digital Signal Processing: COSC390}
\author{Jordan Hanson}
\institute{Whittier College Department of Physics and Astronomy}

\begin{document}
\maketitle

\begin{frame}{Unit 1.2 Outline}
Previous lectures covered:
\begin{itemize}
\item Complex numbers 1: Arithmetic and some calculus (continuous and discete) ... see Chapter 30 of text
\end{itemize}
This lecture will cover:
\begin{itemize}
\item \alert{Complex numbers 2: The Fourier series and Fourier transform (continuous and discrete)}
\end{itemize}
Next lecture will cover:
\begin{itemize}
\item \textit{Time-permitting}: The Laplace transform (continuous and discrete)
\end{itemize}
\end{frame}

\section{Complex numbers 2: theory and examples}

\begin{frame}{Complex numbers 2: theory and examples}
Review: Let's work the following examples.
\begin{enumerate}
\item Let $z_1 = x_1 + j y_1$, and $z_2 = x_2 + j y_2$.  Simplify $z = \frac{z_1^* z_2}{|z_1|^2 + |z_2|^2}$ into real and imaginary parts.
\item Express $z$ in polar form and plot it for $x_1 = y_1 = 1.0$, and $x_2 = y_2 = -1.0$.
\item Express the function $v(t) = v_0 \cos(\omega t + \phi_0)$ as a phasor, and plot it.
\end{enumerate}
\end{frame}

\begin{frame}{Complex numbers 2: theory and examples}
The \alert{\textbf{Fourier series}} representation of a function $f(x)$ is written:
\begin{equation}
S(x) = \frac{A_0}{2}+\sum_{i=1}^{\infty} \left( A_n \cos(nx) + B_n \sin(nx) \right)
\end{equation}
with
\begin{align}
A_n &= \frac{1}{\pi} \int_0^{2\pi} f(x) \cos(nx) dx \\
B_n &= \frac{1}{\pi} \int_0^{2\pi} f(x) \sin(nx) dx
\end{align}
\end{frame}

\begin{frame}{Complex numbers 2: theory and examples}
Let's obtain the \alert{\textbf{Fourier series}} coefficients $A_n$ and $B_n$ for a square-wave signal:
\begin{equation}
f(x) = 1, ~~ 0 \leq x \leq \pi, ~~ 0,  \pi < x \leq 2\pi 
\end{equation}
(Observe on board).  The result: $A_0 = 1.0$, all other $A_n = 0$, odd $B_n$ values follow $2/(n\pi)$, even $B_n = 0$ as well. \\ \vspace{0.5cm}
Create octave code that plots this (see Moodle for example).
\end{frame}

\section{Conclusion}

\end{document}
