\documentclass{beamer}
\usetheme{metropolis}
\usepackage{graphicx}
\usepackage{amsmath}
\title{Digital Signal Processing: COSC360}
\author{Jordan Hanson}
\institute{Whittier College Department of Physics and Astronomy}

\begin{document}
\maketitle

\section{Complex numbers 1: theory and examples}

\begin{frame}{Complex numbers 1}
Convert each of these complex numbers to polar form:
\begin{enumerate}
\item $z = 4 + 4j$ ... $z = \sqrt{32} \exp(j \pi/4)$
\item $z = 1$, $z = j$, $z = -1$, $z = -j$ ... All magitudes 1, the phases are 0 deg, 90 deg, 180 deg, and 270 deg, respectively.
\item In the previous problem, describe in words what is happening to the \textit{phase angle} of each number. ... Rotating by 90 degrees each time.
\end{enumerate}
Convert each of these complex numbers to rectangular form ($z = x + jy$).
\begin{enumerate}
\item $z = 2 \exp(j \pi/4)$ ... $z = \sqrt{2} + j\sqrt{2}$
\item $z = 5 \exp(j \pi)$ ... $z = -5 + 0 j$.
\end{enumerate}
\end{frame}

\begin{frame}{Complex numbers 1}
Suppose that $x_i = 2\pi ft+\phi_i$.  The sum of two sinusoids \textit{in the complex plane} with amplitudes $a_1$ and $a_2$ can then be written
\begin{equation}
V(t) = a_1\exp(j x_1) + a_2\exp(j x_2)
\end{equation}
It is assumed that we would take the real part of $V(t)$ to be physical.
\begin{enumerate}
\item Compute $|V|^2 = V^*V$, and $\phi_2 - \phi_1 = \pi$, $\phi_2 - \phi_1 = 0$.
\item What is $\phi_V = \tan^{-1}(\operatorname{Im}\lbrace V \rbrace/\operatorname{Re}\lbrace V \rbrace)$ in each case?
\end{enumerate}
Why do these results make sense?  Thus, the complex numbers encapsulate the concepts of \textit{constructive} and \textit{destructive} interference.
\end{frame}

\begin{frame}{Complex numbers 1}
\begin{enumerate}
\item $|V|^2 = (a_1 + a_2)^2$ for $\Delta \phi = 0$, and $|V|^2 = (a_1 - a_2)^2$ for $\Delta \phi = \pi$.
\item For the phase, note that 
\begin{equation}
\phi_V = \tan^{-1}\left( \frac{a_1\sin(x_1) + a_2\sin(x_2)}{a_1\cos(x_1) + a_2\cos(x_2)} \right)
\end{equation}
but to simplify, you can take $t = 0$ as an example, along with $a_1 = a_2$.  For $\Delta \phi = 0$, that means
\begin{equation}
\phi_V = \phi_1 = \phi_2
\end{equation}
For the $\Delta \phi - \pi$ case, the functions in the numerator and denominator are out of phase, so they cancel and $\phi_V$ approaches 45 degrees.
\end{enumerate}
\end{frame}

\section{Complex numbers 3: Application to AC cicuits}

\begin{frame}{Complex numbers 1: application to AC circuits}
Recompute $h(\omega)$, but start with $L = 0$ ($Z_2 = 0$).  The answer is
\begin{equation}
h(\omega) = \frac{1}{1+j\omega\tau}
\end{equation}
where $\tau = RC$.  Plotting this function shows that it is about 1 for low frequencies, but decreases to zero for high frequencies.  Frequencies are relative to $1/\tau$.
\end{frame}


\end{document}
