\documentclass{article}
\usepackage{graphicx}
\usepackage[margin=1.5cm]{geometry}
\usepackage{amsmath}

\begin{document}
\twocolumn

\title{Thursday Warm Up, Unit 0: Foundations and Fundamentals}
\author{Prof. Jordan C. Hanson}
\maketitle

\section{Memory Bank}
\small
\begin{itemize}
\item \textbf{Convolution}: this is an operation that characterizes the response $h[n]$ of a linear system.
\begin{equation}
y[i] = h[n] * x[n] = \sum_{j=0}^{M-1}h[j]x[i-j] \label{eq:conv}
\end{equation}
In words, the output at sample $i$ is equal to the produce of the system response $h$ and the input signal $x$, summed over the proceeding $M$ samples (from $j=0$ to $j=M-1$).
\item \textbf{Discrete Delta Function}, $\delta[n]$: A standard impulse response that contains one non-zero sample.  It has the following property:
\begin{equation}
x[n] = \delta[n] * x[n] \label{eq:conv2}
\end{equation}
\item \textbf{Discrete Fourier Transform}, for a sampled, digitized signal $x_n$:
\begin{equation}
X_{\rm k} = \sum_{n=0}^{N-1} x_n e^{-2\pi j (k/N) n}
\end{equation}
\item In DFT analysis, we often need to know the $\Delta t$, time duration for samples, and the sampling rate, $f_{\rm s}$.  Note that $1/f_{\rm s} = \Delta t$.
\item For a sinusoid of frequency $f$ (Hz), the period is $T = 1/f$ (seconds).
\end{itemize}

\section{Unit Conversions for Frequency, Period, and Sampling Rate}

\begin{enumerate}
\item Given the following \textit{sampling rates,} give the time duration of samples:
\begin{itemize}
\item 200 MHz
\item 20 MHz
\item 1.25 GHz
\item 0.750 MHz
\end{itemize}
\item Given the following \textit{time duration of samples,} give the sampling rates:
\begin{itemize}
\item 1 ns
\item 0.25 ns
\item 0.67 $\mu$s
\item 0.33 $\mu$s
\end{itemize}
\end{enumerate}

\section{The Discrete Fourier Transform}

\begin{enumerate}
\item Type \verb+help fft+ in an \verb+octave+ command window.  Read about the various ways to input data into this function that computes the ``fast Fourier transform'' of the data.
\item Write a brief \verb+octave+ script that defines a sampling rate, time samples, and a vector of data representing the product of two sinusoids, one with $f_1 = 1$ MHz, and the other with $f_2 = 1$ kHz.  To this product, add the sinusoid with $f_1$, times an amplitude larger than 1.  The maximum value in the time vector should be at least a hundred milliseconds.  The sampling rate should be more than 2 MHz.
\item Pass the data into the \verb+fft()+ function, and store the output in a variable, \verb+X+.
\item Keep only the first half of the data output, \verb+X = X(1:end/2)+.
\item Let $f_{\rm s}$ be the sampling rate.  Define frequencies as \\ \verb+f = linspace(0,fs/2,N)+, where $N$ is the length of \verb+X+.
\item Multiply the vector \verb+X+ by $1/f_{\rm s} = \Delta t$, then plot the following quantity: 20 times the base-10 logarithm of the absolute value of \verb+X+, versus the frequencies \verb+f+. What do you see?
\item The vertical axis has units of \textbf{decibels} (dB), and the horizontal axis units are Hz.  How can you modify the horizontal axis units to MHz?
\item Why does the spectrum have the structure that it does?
\item Repeat this exercise with gaussian noise added.  The spectrum we are creating is called an \textit{amplitude modulated} (AM) spectrum.
\end{enumerate}

\end{document}
