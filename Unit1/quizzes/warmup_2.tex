\documentclass{article}
\usepackage{graphicx}
\usepackage[margin=1.5cm]{geometry}
\usepackage{amsmath}

\begin{document}
\twocolumn

\title{Thursday Warm Up, Unit 0: Foundations and Fundamentals}
\author{Prof. Jordan C. Hanson}
\maketitle

\section{The Breadth and Depth of DSP}

\begin{enumerate}
\item \textbf{Echo control}: Suppose a sinusoidal signal with frequency $f$ is sent through a communication channel, and causes and echo.  Let $v(t)$ be the original signal, and $w(t)$ be the echo:
\begin{align}
v(t) &= v_0 \sin(2\pi ft) \\
w(t) &= a v_0 \sin(2\pi ft - \phi)
\end{align}
The echo signal $w(t)$ has a phase shift $\phi=\pi/4$, or 45 degrees. (a) As accurately as you can, graph $v(t)$ and $w(t)$ on the same plot, with $a = 0.1$.  (b)  Graph the total signal: $v(t) + w(t)$.  (c) Add one final signal that \textit{cancels} $w(t)$, but \textit{does not cancel} $v(t)$.\footnote{Now imagine doing this for all frequencies, simultaneously.} \\ \vspace{4cm}
\item \textbf{Sampling a sine wave}: Let a set of sample times be $0$, $\Delta t$, $2\Delta t$, ... $n\Delta t$.  The period of $v(t)$ in the previous exercise is $T = 1/f$.  That is, if $f = 4$ kHz, then $T = 1/4$ kHz$^{-1}$. (a) Show that kHz$^{-1}$ is 1 millisecond.  (b) Show that if $f = 10$ kHz, then $T = 0.1$ ms. (c) If we want to sample $v(t)$ and transmit it digitally, then we should probably have $\Delta \ll T$.  Let $f = 5$ kHz for $v(t)$, and $\Delta = 0.02$ ms.  How many samples per period? (d) If instead $\Delta t = 0.002$ ms, how many samples per period? (e) If $\Delta t = 0.002$ ms, what is $1/\Delta t = f_s$, the \textit{sampling frequency}? \\ \vspace{4cm}
\item \textbf{Signal envelopes}: Suppose we send a \textit{gaussian pulse} through a DSP system, with signal shape 
\begin{equation}
p(t) = A e^{-\frac{1}{2}(t-t_0)^2/\sigma^2} \sin(2\pi ft)
\end{equation}
This is a signal with sinusoidal behavior and a gaussian \textit{envelope}.  The envelope is the part of the function multiplying the sine wave.  At $t=t_0$ the envelope has a maximum amplitude of $A\sin(2\pi ft_0)$. (a) Graph the \textit{envelope} of the signal.  (b) Add the sinusoidal oscillation to your graph. (c) If we are going to sample this signal envelope, which of the following should be true:
\begin{itemize}
\item A: $\Delta t = \sigma$
\item B: $\Delta t > \sigma$
\item C: $\Delta t < \sigma$
\item D: $\Delta t \ll \sigma$
\end{itemize} \vspace{4cm}
\item \textbf{Digitizing voltages}: Suppose we are dealing with an AC circuit that produces waveforms for audio systems.  The output runs from -2.5 to 2.5 Volts.  (a) What is the range if we add an offset of +2.5 V to the output signals?  (b) If we can \textit{digitize} the new voltage range into 256 steps, what is the voltage range between steps?
\end{enumerate}

\end{document}
