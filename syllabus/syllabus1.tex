\title{Syllabus for Digital Signal Processing: COSC360}
\author{Dr. Jordan Hanson - Whittier College Dept. of Physics and Astronomy}
\date{\today}
\documentclass[10pt]{article}
\usepackage[a4paper, total={18cm, 27cm}]{geometry}
\usepackage{outlines}
\usepackage{hyperref}

\begin{document}
\maketitle

\begin{abstract}
A broad overview of digital signal processing (DSP) is offered, beginning by reviewing relevant statistics and complex numbers. The linear DSP system is then introduced, followed by the topic of filtering data digitally. The Fourier, Laplace, FFT, and z transforms are presented and applied to digital filtering. The second half of the course involves DSP applications for scientific data. Among the application topics are audio systems and music, human hearing, digital imaging, electrical circuits, and applied neural networks. Free and open-source software packages in the \verb+octave+ programming language are used in this course.  The textbook is also an open educational resource and is freely available to students.
\end{abstract}
\noindent
\textit{\textbf{Pre-requisites}: MATH142} \\
\textit{\textbf{Course credits, Liberal Arts Categorization}: 3 Credits, None} \\
\textit{\textbf{Regular course hours}: Monday through Thursday from 9:00 am - 11:50 pm in SLC 232.  Asynchronous course content will be presented each Friday.} \\
\textit{\textbf{Instructor contact information}: jhanson2@whittier.edu, tel. 562.351.0047} \\
\textit{\textbf{Office hours}: SLC 212, after daily lectures, and via Zoom.  The link to the booking service below may be used to schedule meetings when the instructor is guaranteed to be free.
\begin{itemize}
\item Booking service: \url{https://fgucmvjkylvmgqfsco.10to8.com}
\item Zoom info: Meeting ID: 796 092 0745, passcode: 667725
\end{itemize}
}
\textit{\textbf{Attendance/Absence}: There are 13 3-hour class periods in this January-term course, plus 3 days of asynchronous work.  Attendance is mandatory at all lectures, however there is no penalty if a student must miss a class due to illness or family emergency.} \\
\textit{\textbf{Late work policy}: Late work accepted at the discretion of the professor.} \\
\textit{\textbf{Text}: The Scientist and Engineer's Guide to Digital Signal Processing (www.dspguide.com) - Steven Smith.  This textbook is open-access, free of charge.} \\
\textit{\textbf{Grading}:  See Table \ref{tab:grading} for a listing of graded assignments and the relative weighting.}
\begin{table}
\centering
\begin{tabular}{| c | c | c |}
\hline
Item & Fraction of Grade & Note \\ \hline
Friday quiz 1 & 10\% & Take-home style, open-book, January 7th \\ \hline
Friday quiz 2 & 15\% & Take-home style, open-book, January 14th \\ \hline
Friday quiz 3 & 15\% & Take-home style, open-book, January 21st \\ \hline
Homework sets & 30\% & 2-3 exercises per day \\ \hline
Final Project and Presentation & 20\% & Live or digital (see \textbf{Final Projects} above). \\ \hline
Attendance and participation & 10\% & Example: in-class activities \\ \hline
\end{tabular}
\caption{\label{tab:grading} Grading percentages for DSP.  The take-home quizzes will cover the reading and exercises of the current week, and will include analytic and coding exercises.  Homeworks will usually be due one day after they are assigned, with exceptions.  Students must complete a final project proposal, approved by the professor, before focusing on the research.  As part of the COVID-19 protocols for Whittier College, daily attendance will be taken.}
\end{table} \\
\textit{\textbf{Grade Settings}: 59\% or less: F, 60-69\%: D, 70-79\%: C, 80-89\%: B, 90-100\%: A.  Pluses and minuses: 0-3\% minus, 3\%-5\% straight, 6\%-10\% plus (e.g. 79\% = C+, 91\% = A-)} \\
\textit{\textbf{Homework sets}: The daily homework will typically be 2-3 excercises covering the course material of the same day, and due the following day.  If additional time is required to complete an assignment, please contact the instructor.  The goal is to be flexible while staying on pace.} \\
\textit{\textbf{Final projects}: The final project will involve data brought to the class by the individual students.  The data must be digitized or digital in nature, and be analyzed to establish a result to be presented to the class.  Examples include lab-bench ADC data from a variety of sensors, audio and music, digital images, stock and financial time-series, and weather data.  The final presentation should be longer, approximately 20 minutes.  Students should begin gathering and/or sharing data early on in the course to facilitate creation of the final presentation.  The presentation can be in traditional format or as a digital storytelling project.  For digital storytelling projects, Whittier College has a site license for the online video editing package WeVideo.  A short WeVideo tutorial will be given as part of the course.} \\
\textit{\textbf{ADA Statement on Disability Services}: Whittier College is committed to make learning experiences as accessible as possible.  If you experience physical or academic barriers due to a disability, you are encouraged to contact Student Disability Services (SDS) to discuss options.  To learn more about academic accommodations, email disabilityservices@whittier.edu, call 562.907.4825, or go to SDS which is located on the ground floor of Wardman Library.} \\
\textit{\textbf{Academic Honesty Policy}: \url{http://www.whittier.edu/academics/academichonesty}} \\
\textit{\textbf{Course Objectives}:}
\begin{itemize}
\item Gain understanding of the origin of digital data and how to use it
\item Practice analytic and numerical calculations in digital contexts
\item Write computer code that processes digital data
\item Write and speak comfortably about technical course content
\end{itemize}
\textit{\textbf{Course Outline}:}
\begin{outline}[enumerate]
\1 Unit 1: Statistics and Probability, Complex Numbers, Noise in Digital Systems
\2 Complex numbers, level 1: arithmetic and some calculus
\2 Complex numbers, level 2: The Fourier series and Fourier transform
\2 Complex numbers, level 3: (time permitting) The Laplace transform and z-transform
\2 Statistics and probability: The normal distribution, other useful distributions
\2 Noise: Digitization and sampling
\2 Noise: Spectral properties of noise, ADC and DAC
\1 Unit 2: Linear time-invariant (LTI) systems and filtering
\2 LTI: Definition of LTI systems,convolution and correlation, examples of non-LTI systems
\2 LTI: \textit{Special topic:} Radio-frequency antennas
\2 Filtering: Signals and filtered signals
\2 Filtering: Examples of filters, moving average and window-sync filters
\2 Filtering: Recursive filters
\2 Application of filters: radio thermal noise reduction
\2 Application of filters: CW rejection
\1 Unit 3: Various DSP applications
\2 Audio signals and audio compression
\2 Digital images
\2 Electrical circuits
\2 Applied neural networks (time permitting)
\2 Return to applications of Fourier transform (FFT) and Laplace transform (z-transform) (time permitting)
\1 Final presentations - During the last few days of the course, final projects designed by each student will be shared.
\end{outline}
\textit{\textbf{Important Due Dates}:}
\begin{outline}[enumerate]
\1 Friday take-home quiz 1: January 7th, 2022.  Due Monday, January 10th.  Submitted to instructor via Moodle.
\1 Friday take-home quiz 2: January 14th, 2022.  Due Monday, January 17th.  Submitted to instructor via Moodle.
\1 Friday take-home quiz 3: January 21st, 2022.  Due Monday, January 24th.  Submitted to instructor via Moodle.
\1 Final project proposal: January 17th, 2022.  Given in standard format, speaking in front of class, or as a digital storytelling project.  The final project proposal shouuld be a one-page executive summary of the research to be performed, the hypothesis being tested, data to be collected, and research methods from the course content to be used to achieve the results.
\1 Final project presentations: January 25th and January 26th.  Students should aim to give 20-minute presentations in either live or digital storytelling format.  Presentations will be evaluated on both attention to detail in the analysis, and the overall clarity of the explanation of the results.
\end{outline}
\end{document}
