\title{Syllabus for Digital Signal Processing: COSC390}
\author{Dr. Jordan Hanson - Whittier College Dept. of Physics and Astronomy}
\date{\today}
\documentclass[10pt]{article}
\usepackage[a4paper, total={18cm, 27cm}]{geometry}
\usepackage{outlines}
\usepackage{hyperref}

\begin{document}
\maketitle

\begin{abstract}
A broad presentation of the subject of digital signal processing (DSP) will be offered.  The course will begin with a review of relevant statistics, complex numbers, and types of noise in digital systems.  Next the concept of a linear DSP system and the corresponding mathematical techniques will be introdued.  From there, a broad overview of the topic of digital filters and data processing will be given, proceeding to DSP applications to scientific data analysis.  Among the application topics are audio systems and data compression, electrical circuits, digital imaging, and applied neural networks.  Time permitting, advanced topics in DSP with complex numbers will be covered, including the Fourier and Laplace transforms and their digital counterparts, the FFT and z-transforms.  The text for this course will be open source and therefore free for students.
\end{abstract}
\noindent
\textit{\textbf{Pre-requisites}: Calculus I and Introduction to Computer Science.} \\
\textit{\textbf{Course credits, Liberal Arts Categorization}: 3 Credits, None} \\
\textit{\textbf{Regular course hours}: Monday through Friday from 9:00 am - 11:50 pm in SLC 232} \\
\textit{\textbf{Instructor contact information}: jhanson2@whittier.edu, tel. 562.907.5130} \\
\textit{\textbf{Office hours}: January term office hours by appointment (open-door policy)} \\
\textit{\textbf{Attendance/Absence}: There are 16 3-hour class periods in this January-term course.  Attendance is mandatory for all of them, but attendance and class participation grading will be at the discretion of the professor.} \\
\textit{\textbf{Late work policy}: Late work will not be accepted.} \\
\textit{\textbf{Technology policy}: The use of mobile, tablet, or laptop devices during class is not allowed, with the exception of note-taking.} \\
\textit{\textbf{Text}: The Scientist and Engineer's Guide to Digital Signal Processing (www.dspguide.com) - Steven Smith} \\
\textit{\textbf{Grading}: There will be three tests worth 10\% each, given on Fridays.  Short, daily homework assignments will be worth 30\% of the final grade.  In-class reading quizzes and class-participation will be worth 10\%.  A final project and presentation given towards the end of the course will be worth 20\% of the final grade.  A final exam will be given in-class on January 28th, 2019, and it will be worth 10\% of the final grade.  As this is a fast-paced technical course, partial credit will be awarded, and bonus points will be awarded to those who push themselves to cover extra problems and topics independently.} \\
\textit{\textbf{Grade Settings}: $<60\%$ = F, $>60\%,\leq 70\%$ = D, $>70\%,\leq80\%$ = C, $>80\%,\leq 90\%$ = B, $<90\%,\leq 100\%$ = A.  Pluses and minuses: 0-3\% minus, 3\%-6\% straight, 6\%-10\% plus (e.g. 79\% = C+, 91\% = A-)} \\
\textit{\textbf{Homework sets}: The daily homework will typically be 2-3 excercises covering the course material of the same day, and due the Friday of the week the are assigned.} \\
\textit{\textbf{Final project and presentation}: The final project will involve data brought to the class by the individual students.  The data must be digitized or digital in nature, and be analyzed to establish a result to be presented to the class.  The final presentation should be longer, approximately 20 minutes.  Students should begin gathering and/or sharing data early on in the course to facilitate creation of the final presentation.} \\
\textit{\textbf{ADA Statement on Disability Services}: Whittier College is committed to make learning experiences as accessible as possible. If you experience physical or academic barriers due to a disability, you are encouraged to contact Student Disability Services (SDS) to discuss options. To learn more about academic accommodations, email disabilityservices@whittier.edu, call (562) 907-4825, or go to SDS which is located on the ground floor of Wardman Library.} \\
\textit{\textbf{Academic Honesty Policy}: \url{http://www.whittier.edu/academics/academichonesty}} \\
\textit{\textbf{Course Objectives}:}
\begin{itemize}
\item Gain understanding of the origin of digital data and how to use it
\item Practice analytic and numerical calculations in digital contexts
\item Write computer code that processes digital data
\item Write and speak comfortably about digital concepts
\end{itemize}
\clearpage
\textit{\textbf{Course Outline}:}
\begin{outline}[enumerate]
\1 Unit 1: Statistics and Probability, Complex Numbers, Noise in Digital Systems
\2 Complex numbers, level 1: arithmetic and some calculus
\2 Complex numbers, level 2: The Fourier series and Fourier transform
\2 Complex numbers, level 3: (time permitting) The Laplace transform and z-transform \footnote{We will return to these topics if time allows.}
\2 Statistics and probability: The normal distribution, other useful distributions
\2 Noise: Digitization and sampling
\2 Noise: Spectral properties of noise, ADC and DAC
\2 \textit{Special topic:} computer numbers\footnote{We will return to this topic if time allows.}
\1 Unit 2: Linear time-invariant (LTI) systems and filtering
\2 LTI: Definition of LTI systems,convolution and correlation, examples of non-LTI systems
\2 LTI: \textit{Special topic:} Radio-frequency antennas\footnote{This is an area of expertise relevant to the research of the professor; we may cover it in detail if the class desires.}
\2 Filtering: Signals and filtered signals
\2 Filtering: Examples of filters, moving average and window-sync filters
\2 Filtering: Recursive filters
\2 Application of filters: radio thermal noise reduction
\2 Application of filters: CW rejection
\1 Unit 3: Various DSP applications
\2 Audio signals and audio compression
\2 Digital images
\2 Electrical circuits
\2 Applied neural networks (time permitting)
\2 Return to applications of Fourier transform (FFT) and Laplace transform (z-transform) (time permitting)
\1 Final presentations - During the last few days of this course, we will discuss and share projects designed by the professor/student which will use concepts from the course to solve a particular scientific problem using data found/collected by the student.
\end{outline}
\end{document}
