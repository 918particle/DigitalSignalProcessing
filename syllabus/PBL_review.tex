\title{PBL Summary for Digital Signal Processing: COSC360}
\author{Dr. Jordan Hanson - Whittier College Dept. of Physics and Astronomy}
\date{\today}
\documentclass[10pt]{article}
\usepackage[a4paper, total={18cm, 27cm}]{geometry}
\usepackage{outlines}
\usepackage{hyperref}

\begin{document}
\maketitle

\begin{abstract}
Our course in modern DSP unfolded in several phases.  First, an overview of digital signal processing (DSP) basics was given in Unit 0. Second, DSP filtering techniques were covered in Unit 1.  Third, DSP applications to audio systems, digital images, electrical circuits, and neural networks were introduced in Unit 2.  Finally, and synchronously with Unit 2, we began a project-based learning (PBL) environment to build modular synthesizer circuits with DSP applications.  We will gather data in this document to assess the PBL learning environment.
\end{abstract}
\noindent
\textit{\textbf{PBL environment objective}: To understand, build and test modern audio circuits using standard electronics prototyping equipment.  These included breadboards and power supplies, DVMs, signal generators, and oscilloscopes.} \\
\textit{\textbf{Format}: Students worked in pairs to make progress toward the objective.} \\
\textit{\textbf{Course time}: Approximately 30\% of the course was devoted to working on the PBL projects.} \\
\textit{\textbf{Materials}: The Moritz Klein/Erica Synths EDU kits, and prototyping equipment, were provided free of charge.} \\
\textit{\textbf{Grading}: See Tab. \ref{tab:grading}.}
\begin{table}[hb]
\centering
\begin{tabular}{| c | c | c |}
\hline
\textbf{Assignment} & \textbf{Weight} & \textbf{Target Date} \\ \hline
PBL Individual & 20\% & April 28th - May 2nd, 2024 (in-class) \\ \hline
PBL Team & 10\% & April 28th - May 2nd, 2024 (in-class) \\ \hline
\end{tabular}
\caption{\label{tab:grading} Itemized grade components from the PBL learning environment.}
\end{table} \\
\textit{\textbf{Questions for Students}:}
\begin{itemize}
\item List the top three ideas or techniques you learned from the PBL learning environment: \\ \vspace{0.5cm}
\item List any ideas or techniques you wish we had reached or included in the PBL learning environment: \\ \vspace{0.5cm}
\item How well do you feel we have met the overall objective of \textit{understanding} the audio circuit we have constructed? \\ \vspace{0.5cm}
\item How well do you feel we have met the overall objective of \textit{building} the audio circuit we have constructed? \\ \vspace{0.5cm}
\item How well do you feel we have met the overall objective of \textit{testing} the audio circuit we have constructed? \\ \vspace{0.5cm}
\item On a scale of 0-20\%, what PBL individual score\footnote{Note: your opinion is important to us as instructors, and we will take it into account.} do you feel is warranted for your progress this semester? \\ \vspace{0.25cm}
\item On a scale of 0-10\%, what PBL team score\footnote{Note: your goal here is to assess the participation of your lab partner in the team effort.} do you feel is warranted for the progress of your team this semester? \\ \vspace{0.25cm}
\item Finally, would you have liked to spend \textit{more time} on the PBL learning environment, or \textit{less time} on it?
\end{itemize} \vspace{1cm}
\textbf{Name/ID:}\underline{$~~~~~~~~~~~~~~~~~~~~~~~~~~~~~~~~~~~~~~~~~~~~~~~~~~$}\textbf{Name/ID of lab partner:}\underline{$~~~~~~~~~~~~~~~~~~~~~~~~~~~~~~~~~~~~~~~~~~~~~~~~~~$}
\end{document}